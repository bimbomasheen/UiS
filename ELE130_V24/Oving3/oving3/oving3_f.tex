\runningheader{Oppgave f)}{}{Side \thepage\ av \numpages}

\item[] I de neste to deloppgavene \ref{oppg:f})-\ref{oppg:g}) skal du
  jobbe med følgende enkle signal
\begin{equation}
  \label{eq:u2}
 \boxed{ u(t) = 1 }
\end{equation}
Hensikten med å bruke et såpass enkelt signal er å demonstrere 
noen viktige egenskaper ved numerisk integrasjon og derivasjon.


% ********************************************************
% oppgave f) 
% ********************************************************  
\item
{\bf Numerisk integrasjon av et signal som er forsterket}
\label{oppg:f}

I denne oppgaven skal du beregne følgende integral 
\begin{equation}
  \label{eq:4aa}
  y(t) = K_{i}  \int_{0}^{t}  u(t)dt
\end{equation}
ved å bruke  funksjonen {\tt EulerForover} fra deloppgave~\ref{oppg:d}). 
Parameteren $K_{i}$ er en forsterkningsfaktor som kan være et vilkårlig tall,
positivt eller negativt. 
I koden nedenfor er $K_{i}{=}0.4$, og det er implementert to forskjellig varianter av
numerisk integrasjon. Den ene varianten er riktig, den
andre er feil, og målet med oppgaven er at du skal finne ut hvilken
som er riktig, og forklare hvorfor det er slik.
\vspace*{-5mm}
\begin{lstlisting}[caption={Kode for oppgaven.},
language= Matlab, label=kode:1f, numbers=none] 
% initialverdi
y1(1) = 0;
y2(1) = 0;

Ki = 0.4;
for k = ..
    y1(k) =    EulerForover(y1(k-1), T_s, Ki*u(k-1));
    y2(k) = Ki*EulerForover(y2(k-1), T_s,    u(k-1));
end
\end{lstlisting}
\begin{itemize}
\item Pass på at \fbox{\tt Ki = 0.4;} og kjør koden. Kan du allerede
  nå ut fra kurvene finne ut om det er {\tt y\_1} eller {\tt y\_2} som er riktig?

\item Endre parameteren $K_{i}$ til \fbox{\tt Ki = 1.4;} og kjør
  koden på ny. Blir du noe klokere av det? 

\item For å bli sikker på svaret er det smart å legge inn følgende kode
  i linjen etter at {\tt u} blir definert.  

\begin{lstlisting}[numbers=none]
u(5:end) = 0;  % setter siste del av u som 0
\end{lstlisting}

Du kan nå endre $K_{i}$ mellom \fbox{\tt Ki = 0.4;} og \fbox{\tt
  Ki = 1.4;}, og du vil forhåpetligvis oppdage hvilken som er den
riktige.

\item Hva er den egentlige grunnen til at den som er feil, er feil.
  
\end{itemize}
