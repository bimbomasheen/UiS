\runningheader{Oppgave k), frivillig}{}{Side \thepage\ av \numpages}
% ********************************************************
% oppgave k) 
% ******************************************************** 
\item \label{pkt:FIR_IIR}
  {\bf Filtrering av ulike testsignal}
  
I denne oppgaven skal du bruke filterfunksjonene fra de forrige
deloppgavene til å  filtrere følgende fire testsignal (formulert tidskontinuerlig):
\begin{align}
\text{impuls:} \quad  x_{1}(t) & =
      \begin{cases}%
  0 &\text{for } t<10\\
  1 &\text{for } t=10\\
   0& \text{for } t >10
 \end{cases}\\
\text{firkantpuls:} \quad  x_{2}(t) & =
      \begin{cases}%
  0 &\text{for } t<10\\
  1 &\text{for } 10\leq t \leq 20\\
   0& \text{for } t > 20
 \end{cases}\\
\text{sinussignal:} \quad   x_{3}(t) & = 0.1 \sin(2\pi 0.1 t)+0.2 \label{eq:3}\\
  \text{sinussignal med støy:} \quad   x_{4}(t) &
                                                  = 0.1 \sin(2\pi 0.1 t)+0.2 +w(t)\label{eq:3a}  
\end{align}
I skallfilen
er den diskret tidsvektoren
\fbox{\tt t}  basert på en samplingsfrekvens på
  \begin{equation}
    \label{eq:1511}
      f_{s} = 2~\text{Hz}
    \end{equation}
    som gir en samplingstid på $T_{s}{=}0.5$ sekund.
    Siden Nyquistfrekvensen ved denne samplingsfrekvensen er
    \begin{equation}
      \label{eq:1_fs}
      f_{N}= 0.5{\cdot} f_{s} = 1~\text{Hz}
    \end{equation}
    så betyr det at vi sampler hurtig nok til å få med oss informasjonen fra
    sinussignalene gitt i ligningene~\eqref{eq:3} og \eqref{eq:3a},
    siden begge har en svingefrekvensen på $f{=}0.1$ Hz. 
    Videre er det i skallfilen gitt kode for de fire signalene
    \fbox{\tt x1} til  \fbox{\tt  x4} hvor bruker vi sammenhengen 
  \begin{equation}
    \label{eq:7}
    k{\cdot}T_{s} = t_{k}
  \end{equation}
  for å bestemme hvilken indeks $k$ som hører til et gitt tidspunkt.
  Siden vi jobber med Matlabindekser legger
  vi til 1 i beregningen. Dersom du velger et tidsskritt som ikke gir
  heltallsverdier av $k$, så må du bruke {\tt round}-funksjonen som
  \fbox{\tt k = round(t\_impuls/T\_s+1);}.

  Siden koden for FIR- og IIR-filteret er basert på variabelen {\tt x}, 
  og ikke {\tt x1}, {\tt x2}, {\tt x3} eller {\tt x4} som vi nettopp
  har definert, så velger du  hvilket signal du vil bruke der
  hvor det i skallfilen står:
  
\begin{lstlisting}[caption={Syntaks for å velge hvilket signal som
    skal filtreres.},  language= Matlab,   label=kode:velg,
numbers=none] 
%------------------------------------------------------
% Velg hvilket signal som skal filtreres. 
x = x1;
% For x1 og x2 er det smart å bruke kurveform = ':x';
% For x3 og x4 er det smart å bruke kurveform = '-';
kurveform = '-';
%------------------------------------------------------
\end{lstlisting}

  
Ferdigstill det som mangler i skallfilen, og gjør deretter følgende
oppgaver, hvor du for hver oppgave tar med resultatfigurene i innleveringen din.

\subsubsection*{\bf Filtrering av impulsen $x_{1}(t)$}

  Hensikten med dette signalet er
  å demonstrer hvorfor det heter FIR og IIR. Det betyr at du vil se at
  impulsen gir en  respons som blir 0 etter en viss tid for
  FIR-filteret ({\it finite}),  og en respons som aldri blir 0 for
  IIR-filteret ({\it infinite}). 

  \begin{itemize}
  \item Bruk \fbox{\tt M = 10}  og  \fbox{\tt alfa = 0.3} og kjør
    koden. Trykk inn på
    de røde og grønne linjene til hhv. FIR- og IIR-filteret rundt
    $t{\approx}30$ sekund og les av verdien. Ta med figuren i
    innleveringen din. 
  \item Er det logisk at maksimalverdien av  $y_{k,FIR}$  er 0.1 
    når \fbox{\tt M = 10}  mens maksimalverdien av   $y_{k,IIR}$ er 0.3 når
    \fbox{\tt alfa = 0.3}? Begrunn svaret.
  \end{itemize}

\subsubsection*{\bf Filtrering av firkantpulsen $x_{2}(t)$}

  Hensikten med dette signalet er å
  demonstrere {\it dynamikken} til filteret, altså hvor raskt det
  svinger seg inn til ny stasjonær verdi gitt at inngangen endres som
  et sprang. Husk at det positive spranget går ved
  $t_{\text{sprang}}{=}10$~sekund, og det negative spranget går ved
  $t_{\text{sprang}}{=}20$~sekund.
  
 
  \begin{itemize}
  \item Bruk \fbox{\tt M = 10}  og  \fbox{\tt alfa = 0.3} og kjør
    koden. Forklar med utgangspunkt i ligningen for FIR-filteret
    hvorfor responsen stiger som en lineær kurve før den
    til slutt flater ut.

  \item Bruk sammenhengen
    \begin{equation}
      M =\frac{t_{\text{FIR}}}{T_{s}}+ 1
    \end{equation}
    til å anslå filtervinduets lengde $t_{\text{FIR}}$~[s].  Hvordan
    relateres verdien av $t_{\text{FIR}}$ til responsen i FIR-filteret?
    
  \item Bruk sammenhengen
    \begin{equation}
  \label{eq:12ny}
  \boxed{\alpha = \frac{T_{s}}{\tau + T_{s}} }
\end{equation}
til å anslo  tidskonstanten $\tau$ for IIR-filteret. Hvordan
relateres verdien av $\tau$ til responsen i IIR-filteret?
   

  \item Hvordan endres dynamikken (hurtigheten) til IIR-filteret og
    selve responsen til FIR-filteret dersom du endrer til
    \fbox{\tt M = 15} og  \fbox{\tt alfa = 0.15}?

\end{itemize}

\newpage

\subsubsection*{\bf Filtrering av sinussignalet $x_{3}(t)$}

  Hensikten med dette signalet er å
  vise hvordan sinussignalet endres gjennom filteret, altså hvor mye
  amplituden blir redusert og hvor stor faseforskyvning er som
  funksjon av filterparametrene.   I denne oppgaven
  kan du med fordel endre kurveformen til \fbox{\tt '-'} slik 
  at kurvene blir ``renere''. 
 Gitt at resposnene i FIR- og IIR-filteret kan formuleres
    tidskontinuerlig som 
    \begin{align}
      \label{eq:11}
      y_{FIR}(t) & = Y_{FIR}{\cdot} \sin(2\pi f t + \varphi_{FIR}) \\
      y_{IIR}(t) & = Y_{IIR} {\cdot}\sin(2\pi f t + \varphi_{IIR}) 
    \end{align}

  \begin{itemize}
  \item  Bruk \fbox{\tt M = 15}  og  \fbox{\tt alfa = 0.15} og kjør
    koden. Hvor stor er amplitudene $Y_{FIR}$ og $Y_{IIR}$?
  \item Hvor stor er faseforskyvningen mellom
    innsignalet $x_{3}(t)$ og de to filtrerte signalene. Det vil si,
    bestem $\varphi_{FIR}^{\circ}$ og
    $\varphi_{IIR}^{\circ}$. (oppgitt i grader) 
  
  \item Øk samplingsfrekvensen $f_s$ til \fbox{\tt f\_s = 4}.
    Hvordan påvirker dette filtreringen? Hva er forklaringen
  på det som skjer? 
\item Behold \fbox{\tt f\_s = 4}. Hva må du gjøre med
  $M$ og $\alpha$ for å få et resultat som ligner på det du fikk når
  \fbox{\tt f\_s = 2}?
  
\item Sett \fbox{\tt M = k} inne i
  for-løkken, og forklar hva som skjer i FIR-filteret.


\end{itemize}

\subsubsection*{\bf Filtrering av sinussignalet med støy $x_{4}(t)$}
   Hensikten med dette signalet er å vise at hvordan
  du kan filtrere bort støyen og helst sitte igjen med sinussignalet som
  ligger i bunn.   

  \begin{itemize}
  \item 
  Behold \fbox{\tt f\_s = 4} og finn de $M$ og $\alpha$-verdiene som
  du synes fjerner støy på best mulig måte samtidig som selve
  sinussignalet beholdes i  størst mulig grad. Her er det ingen feil svar.
  \end{itemize}
  
