\runningheader{Oppgave h)}{}{Side \thepage\ av \numpages}
% ********************************************************
% oppgave h) 
% ******************************************************** 
\item
  {\bf Fenomenet nedfolding}

\label{oppg:h}

  I denne oppgaven tar vi utgangspunkt i hvordan sampling blir utført
  som vist i kapittel 7.2 i kompendiet. 
  Hensikten med oppgaven er å demonstrere hvilken betydningen
  samplingsfrekvensen  $f_{s}$~[Hz] har ved sampling av signaler som
  består av forskjellige frekvenser $f$~[Hz].  Vi skal derfor bruke
  følgende 5 cosinussignal
\begin{align}
  x_{1}(t) & =   \cos(2\pi f_{1} {\cdot}t)  \label{eq:U1} \\
  x_{2}(t) & =   \cos(2 \pi f_{2} {\cdot}t)  \label{eq:U2} \\
  x_{3}(t) & =   \cos(2\pi f_{3} {\cdot}t)  \label{eq:U3}\\
  x_{4}(t) & =   \cos(2\pi f_{4} {\cdot}t)  \label{eq:U4}\\
  x_{5}(t) & =   \cos(2\pi f_{5} {\cdot}t)  \label{eq:U5}
\end{align}
  hvor vi benytter følgende frekvenser 
\begin{equation}
  f_{1} = 0.1,  ~ f_{2}= 0.3, ~ f_{3}= 0.5, ~ f_{4}= 0.9 ~\text{og}~f_{5}= 1.3
  ~\text{Hz} \notag
\end{equation}
Som du ser i skallfilen lager vi først 
den ``kontinuerlige'' tidsvektoren \fbox{\tt tid}  ved å benytte et
veldig kort tidsskritt \fbox{\tt delta\_t = 0.001}.

Deretter lages  vektoren \fbox{\tt t} med samplingstidspunkt som
representerer de tidspunktene hvor målingene blir tatt. Basert på
tidsvektorene {\tt tid} og {\tt t} lages det deretter  
en {\it analog} og en {\it diskret} versjon av alle fem signalene
\fbox{\tt x\_1} til \fbox{\tt x\_5}. Til slutt summeres alle de analoge signalene til
signalet \fbox{\tt x\_a} og alle de diskret signalene til signalet
\fbox{\tt x\_d}. På denne måten kan du kan se effekten av
samplingsfrekvensen \fbox{\tt f\_s} på et sammensatt 
og mer reelt signal (ikke bare et rent cosinussignal).
%{\color{red}Svar på følgende spørsmål:}
\begin{itemize}
\item 
 Hvor stor er samplingsperioden $T_{s}$ for den valgte  
  samplingsfrekvensen $f_{s}{=}0.6$~Hz?
\item 
 Hvor mange elementer er det i signalet \fbox{\tt x\_a} og
 i signalet \fbox{\tt x\_d}?
\item 
 Forklar hva figuren viser? 
\item 
  Hvilke signal ser det ut som vi sampler i \fbox{\tt subplot(7,1,2)} til \fbox{\tt subplot(7,1,5)}?
  Avles periodetidene $T_{p}$ fra de blå stiplede kurvene og beregn
  de tilsynelatende frekvensene for de diskret signalene.
  Sammenlign med frekvensene for de 
  analoge signalene $x_{2}(t)$ til $x_{5}(t)$. 
\item 
 Hva skjer dersom du spesifserer samplingsfrekvensen som
 $f_{s}{=}1$~Hz? 

\item 
  Hva skjer dersom du spesifserer samplingsfrekvensen som
  $f_{s}{=}1.8$~Hz? 

\item 
   Hva skjer dersom du spesifserer samplingsfrekvensen som
   $f_{s}{=}2.6$~Hz? 

\item  Forklar hva som skjer når samplingsfrekvensen øker.



\end{itemize}
