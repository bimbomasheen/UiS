\section*{Om øvingen og innleveringen}

\begin{boxedminipage}{\textwidth}
    \begin{itemize}
      \setlength\itemsep{0mm}
    \item For å bli kjent med prinsippet bak numerisk integrasjon,
      numerisk derivasjon og filtrering, 
      så er alle oppgavene i denne øvingen svært nyttige,
      og de vil fungere som et oppslagsverk for deg når du gjør senere
      øvinger.


  \item For å tilrettelegge for at du skal gjøre flest mulig oppgaver,
    så kan du laste ned og pakke opp \fbox{\tt oving3\_skallfiler.zip}.
    I denne zip-filen vil finne
    \begin{itemize}
    \item  skallfilen \fbox{\tt oving3\_skallfil.m}
      som inneholder uferdig kode for alle oppgavene i denne øvingen
    \item  skallfiler for alle funksjonene som du skal ferdigstille
    \item simulink-skallfilen \fbox{\tt oving3\_a\_m\_skallfil.slx}
      for oppgavene 3a) og 3m)
    \end{itemize}
    Mange av oppgavene virker kanskje omfattende, men svært
    mye av ``infrastruktur-koden'' er allerede
    gitt.

 \item   På tilsvarende måte som i øving 1 bør du bruke trikset
    med å først kommentere hele skriptet \fbox{\tt oving3\_skallfil.m}
    for deretter å ta bort
    kommentarene celle for celle. 

    \item {\bf Hensikten med   øvingen er du skal bruke
      oppgavene som et verktøy til å gi deg forståelse og innsikt for
      de forskjellige temaene/begrepene.}


  \item   {\color{red}  For å få øvingen godkjent må du minst
      gjøre følgende oppgaver:\\
      \fbox{b), d), e), f), h), i), j) }}

      \item   Oppgavene   \fbox{a), c), g), k), l), m), n) o) }  er
        også lærerike og nyttige, men frivillige. Dette er markert i toppteksten.

\item   {\color{red}  For å ta eksamen i emnet  ELE130 så må denne øving være
  godkjent. Husk at øvingene må leveres individuelt.}

  \item Basis for denne øvingen er
    {\color{blue} kapittel 7, 8 og 9} i kompendiet.

    \item På samme måte som i øving 1 og 2 finner du en {\LaTeX}-mal i filen 
    \fbox{\tt oving3.zip}.

  \end{itemize}
  \end{boxedminipage}

  
  



