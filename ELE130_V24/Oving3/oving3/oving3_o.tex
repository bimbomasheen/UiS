\runningheader{Oppgave o), frivillig}{}{Side \thepage\ av \numpages}

% ********************************************************
% oppgave o) 
% ******************************************************** 
\item   {\bf Demonstrasjon av oppsampling}

  Oppsampling  (eng: {\it upsamling}) er en teknikk for å rekonstruere
  samplede signaler.   I denne
  oppgaven skal vi demonstrere dette for en tenkt problemstilling hvor
  du har en motor med en såkalt encoder som kan måle vinkelposisjonen
  til motoren. Anta
  at motoren roterer ca.\ 10 runder, og at du sampler
  vinkelposisjonen for hver {\tt delta\_phi=260}$^{\circ}$, og at
  dette skjer med et 
  tidsskrittet på {\tt T\_s=0.2}~sekund. Dette gir en vinkelhastighet på
  \begin{equation}
\omega_{d} = \frac{260^{\circ}}{0.2}= 1300^{\circ}/\text{s}    
\end{equation}
hvor subskript $d$ impliserer ``degrees''. 
Ved å beregne og plotte
  \begin{equation}
y(t) = \sin(\omega_{d}{\cdot} t)    
  \end{equation}
så kan du  få et visuelt bilde på rotasjonen som har
foregått.  Vær klar over at sinus-funksjonen forventer
en vinkel oppgitt i radianer og ikke grader som over, og at vi strengt
tatt burde skrevet
  \begin{equation}
y(t) = \sin\Bigl(\omega_{d}{\cdot} t {\cdot} \frac{2\pi}{360} \Bigl)    
  \end{equation}
Når det er sagt, så har Matlab en  {\tt sind()}-funksjon som tar
grader som argument.


Som du vil se fra kurven øverst til høyre når du kjører koden,
så er problemet at vi sampler såpass sjelden at det ser ut 
som motoren kun har rotert et par runder. Etter at du trykker OK, vil
de se at vi legger     
inn fiktive, men egentlig høyst reelle, vinkelmålinger og tidspunkt
mellom de samplede  målingene. På den måten utvider vi datasettet
vårt, og grunnen til at vi vet at dette går fint er at motoren har
rotert jevnt i én retning. 

Matlabfunksjonen for å oppsample er
\fbox{\tt interp} som interpolerer 
verdier mellom de avleste vinkelposisjonsmålingene.

Lek gjerne med verdien på {\tt delta\_phi}, fra 30$^{\circ}$ og
oppover. Ved hvilken verdi går det galt?

Som innlevering kan du ta med de signalene du har lekt med og samtidig
forklare hva som skjer.
